\chapter{误差}
\section{真值与误差}
\entry 有测量就会有误差。通常,将某数学量、物理量的\key{真值}记为$x$(不加任何修饰符),而将测量或计算所得的$x$的\key{近似值}记作$\tilde{x}$。

\entry 两种误差:$\begin{cases}\Delta x=x-\tilde{x} \text{\ (绝对误差)}\\ \delta x=\frac{x-\tilde{x}}{x}\text{\ (相对误差)}\end{cases}$

\entry 两种误差限:$\begin{cases}|\Delta x|\leq\varepsilon\text{\ (绝对误差限)}\\|\delta x|\leq\varepsilon_r\text{\ (相对误差限)}\end{cases}$

\entry 相对误差较小时,有近似计算式\footnote{在估计误差时,真值$x$往往难以确定,但绝对误差$|\Delta x|$或绝对误差限$\varepsilon$往往能够确定下来。}
:$|\delta x|\approx\frac{|x-\tilde{x}|}{\tilde{x}}=\frac{\Delta x}{\tilde{x}}\leq\frac{|\varepsilon|}{\tilde{x}}$

\entry 若$|\Delta x|=|x-\tilde{x}|\leq0.5\times10^{-n}$,则称$x$的近似值$\tilde{x}$ \emph{准确到第 $n$ 位小数}。

\example 设$x=0.31682$,则$\tilde{x}_1=0.3$精确到$1$位有效数字,$\tilde{x}_2=0.32$精确到$2$位,$\tilde{x}_3=0.317$精确到$3$位,$\tilde{x}_4=0.3168$精确到$4$位。若取$\tilde{x}_5=0.3169$为$x$的近似值,则其仅精确到$3$位小数。

\section{浮点运算与浮点数集}
\entry 在计算机中,实数将被储存为\key{浮点数},故计算机中的实数运算常被称作\key{浮点运算}。为此,有下面的一些概念与理论。

\entry \key{浮点运算量}:记\emph{一次加法和一次乘法}(如$a+b\times c$)所需的时间为一个\key{时间单位},记为flop。

\example 设$\mathbf{A}_1$为一$10\times20$的矩阵,$\mathbf{A}_2$为一$20\times50$的矩阵,欲计算$\mathbf{A}_1\cdot\mathbf{A}_2$,则运算量为$10\times20\times50=10000$ flop
\footnote{$\mathbf{A}_1$的行乘以$\mathbf{A}_2$的列,每次的浮点运算量为 $20$ flop,总计 $10\times50$ 种组合。}。

\entry \key{浮点数集}:在10进制中,浮点数$\tilde{x}$(或一实数$x$的近似$t$位有效数字的浮点数$\tilde{x}$)可表示如下:
\[fl(x)=\tilde{x}=\pm\left\{\frac{x_1}{10}+\frac{x_2}{10^2}+\frac{x_3}{10^3}+\cdots+\frac{x_t}{10^t}\right\}\times10^l\ (\tilde{x}=0.x_1x_2x_3\cdots x_t\times10^l)\]
其中$1\leq x_1<10$,$0\leq x_j<10$,$j=2,3,\cdots,t$。类似的,在$\beta$进制中,一个数的表示方式:
\[fl(x)=\tilde{x}=\pm\left\{\frac{x_1}{\beta}+\frac{x_2}{\beta^2}+\cdots+\frac{x_t}{\beta^t}\right\}\times\beta^l\]
其中$1\leq x_1<\beta$,$0\leq x_j<\beta$,$j=2,3,\cdots,t$。$\beta^l$称为指数部分,指数$l$满足$L\leq l\leq U$,$L$与$U$分别为下界与上界;$0.x_1x_2\cdots x_t$称为尾数。$fl(x)$称为一个\key{规格化浮点数}。

\entry 称计算机中所能表示的全体数的集合称为\key{浮点数集},记为$F(\beta,t,L,U)$。
\begin{equation}
F(\beta,t,L,U)=\{0\}\cup\left\{\pm\left(\frac{x_1}{\beta}+\frac{x_2}{\beta^2}+\cdots+\frac{x_t}{\beta^t}\right)\times\beta^l:L\leq l\leq U\right\}
\end{equation}

\example \textsf{C++}里的 \verb|float|:4 字节、32 位,如图 \ref{1-2} 所示。可以将这一浮点数集记为$F(2,23,-128,127)$。
\begin{figure}[htbp]
\small\centering
\begin{tikzpicture}[scale=.8]
  \draw (0, 0) rectangle (1, 1);
  \draw (1, 0) rectangle (2, 1);
  \draw (2, 0) rectangle (3, 1);
  \draw[dashed] (3, 0) rectangle (4, 1);
  \draw (4, 0) rectangle (5, 1);
  \draw (5, 0) rectangle (6, 1);
  \draw[dashed] (6, 0) rectangle (7, 1);
  \draw (7, 0) rectangle (8, 1);
  \draw (8, 0) rectangle (9, 1);
  \node at (0.5, 0.5) {$0$}; \node at (0.5, 1.3) {$31$};
  \node at (1.5, 0.5) {$0$}; \node at (1.5, 1.3) {$30$};
  \node at (2.5, 0.5) {$1$}; \node at (2.5, 1.3) {$29$};
  \node at (3.5, 0.5) {\ldots};
  \node at (4.5, 0.5) {$0$}; \node at (4.5, 1.3) {$23$};
  \node at (5.5, 0.5) {$0$}; \node at (5.5, 1.3) {$22$};
  \node at (6.5, 0.5) {\ldots};
  \node at (7.5, 0.5) {$1$}; \node at (7.5, 1.3) {$1$};
  \node at (8.5, 0.5) {$1$}; \node at (8.5, 1.3) {$0$};
  \node at (0.5, -0.3) {$\downarrow$};
  \node at (0.5, -0.8) {底数符号};
  \draw (3, -0.3) node [xscale=3.5, rotate=90] {$\Biggl\{$};
  \node at (3, -0.8) {指数位(含符号)};
  \draw (7, -0.3) node [xscale=3.5, rotate=90] {$\Biggl\{$};
  \node at (7, -0.8) {底数位}; 
\end{tikzpicture}
\caption{\textsf{C++}中\texttt{float}类型变量的储存原理}\label{1-2}
\end{figure}

\entry 浮点数集中的数的个数:$N=2\cdot(\beta-1)\cdot\beta^{t-1}\cdot(U-L+1)+1$

\entry 浮点数$fl(x)$与对应真值$x$的误差:
\begin{itemize}\tl
    \item 绝对误差:$|x-fl(x)|\leq\dfrac12\beta^{-t}\times\beta^l=\dfrac12\beta^{l-t}$
    \item 相对误差:由$|x|\geq0.1\times\beta^l$,有$\dfrac{|x-fl(x)|}{|x|}\leq\dfrac{\beta^{l-t}/2}{\beta^{l-1}}=\dfrac12\beta^{1-t}$
\end{itemize}
此类误差称为\key{舍入误差}。

\entry 计算结果的错误/误差:
\begin{enumerate}\tl
    \item $l\notin[L,U]$:\key{上溢}($l\geq L$)会出错,\key{下溢}($l\leq U$)变为$0$。
    \item 尾数多于$t$位:自动进行舍入处理,造成误差
    \item 有效数字丢失:「\emph{大数吃小数}」
\end{enumerate}


\example 设计算时保留4位有效数字,则$1234+0.3678=1234.3678\approx1234$,在此发生了「大数吃小数」的现象。

\entry 设计数值计算的算法时,应结合浮点数具有的特性,避免上面所提到的各类计算错误。为此,提出以下几条\emph{浮点运算原则}:
\begin{enumerate}\tl
    \item 避免产生大结果的运算,避免小数作为除数。
    \item 避免「大」、「小」数相加减,防止大数吃小数。
    \item 避免相近数直接相减,防止有效数字损失。
    \item 简化运算步骤,减少运算次数
    \footnote{由此避免各类误差的逐次累计。——编者注}。
\end{enumerate}
若原有的计算公式不符合以上的这些原则,则可以通过对原式的等价变换或近似处理,使之符合上面的原则。

\example 设$|x|\ll1$,则可改写数值计算公式$\ln\dfrac{1-\sqrt{1-x^2}}{|x|}$为以下形式,以避免小数作分母:
\[\ln\frac{1-\sqrt{1-x^2}}{|x|}=\ln\frac{x^2}{|x|\cdot(1+\sqrt{1-x^2})}=\ln\frac{|x|}{1+\sqrt{1-x^2}}\]

\section{计算方法的研究内容}
\entry 计算方法课程,并不仅仅包含各类数值计算方法。归结而言,计算方法课程的研究内容可以归纳为:
\begin{enumerate}\tl
    \item 某一问题的数值计算算法(即通常意义上的「计算方法」);
    \item 这些算法的误差、复杂性或收敛速度之估计。
\end{enumerate}
后者至关重要。对于算法的误差或复杂性分析,使这门课程区别于一般的工具性课程。

\entry 针对一些模型,还存在着一类特定的问题,即\textbf{病态问题}。为度量这类问题的性质,需要用到\key{条件数}。
\begin{itemize}\tl
    \item 根据输入数据的微小变化能引起问题之解变化的大小程度,可以将数值计算问题区别为两类:若由此能引起解的很大变化,则称问题是\textbf{病态}的;否则,称一个问题是\textbf{良态}的。病态问题不易精确求解。
    \item \key{条件数}:输入数据$x,\tilde{x}$,输出$f(x),f(\tilde{x})$。设$x\neq0$,$f(x)\neq0$,若存在$m>0$使:
    \[\frac{|f(x)-f(\tilde{x})|}{|f(x)|}\leq m\cdot\frac{|x-\tilde{x}|}{|x|}\ (\text{输出误差}\leq m\cdot\text{输入误差})\]
    则将$m$称为该问题的\textbf{条件数},记为$\cond(f)$。
\end{itemize}

\example $y=\varphi(x_1,x_2,\cdots,x_n)$。输入为$\tilde{x}_1,\tilde{x}_2,\cdots,\tilde{x}_n$,近似解$\tilde{y}=\varphi(\tilde{x}_1,\tilde{x}_2,\cdots,\tilde{x}_n)$,则有
\begin{gather}
\Delta y=\varphi(x_1,x_2,\cdots,x_n)-\varphi(\tilde{x}_1,\tilde{x}_2,\cdots,\tilde{x}_n)
\approx\sum_{i=1}^n\frac{\partial\varphi(\tilde{x}_1,\tilde{x}_2,\cdots,\tilde{x}_n)}{\partial x_i}\Delta x_i\\
\delta y=\frac{\Delta y}{y}\approx\sum_{i=1}^n\frac{\partial\varphi(\tilde{x}_1,\tilde{x}_2,\cdots,\tilde{x}_n)}{\partial x_i}\cdot\frac{\Delta x_i}{y}
=\sum_{i=1}^n\frac{\partial\varphi(\tilde{x}_1,\tilde{x}_2,\cdots,\tilde{x}_n)}{\partial x_i}\cdot\frac{x_i}{y}\cdot\delta x_i
\end{gather}
故可见$\left|\frac{\partial\varphi(\tilde{x}_1,\tilde{x}_2,\cdots,\tilde{x}_n)}{\partial x_i}\cdot\frac{x_i}{y}\right|$即条件数。

\entry \key{稳定性}(数值稳定性):运算中\textbf{舍入误差积累}是否影响结果的可靠性。

\example 欲用数值计算方法求解由$I_k=\e^{-1}\int_0^1x^k\e^x\di x,\ k=0,1,\cdots,7$所定义的一系列定积分的值。
\begin{itemize}\tl
    \item 算法1:构建递推公式
    \begin{equation}\label{1-e1}
    \left\{
    \begin{aligned}
    I_0&=\e^{-1}\int_0^1\di x=1-\dfrac1\e\\
    I_k&=\e^{-1}\int_0^1x^k\e^x\di x=1-kI_{k-1}
    \end{aligned}
    \right.
    \end{equation}
    利用递推关系依次计算$I_0\rightarrow I_1\rightarrow I_1\rightarrow\cdots\rightarrow I_7$。
    \item 算法2:近似计算$I_7$,利用递推关系
    \footnote{即将(\ref{1-e1})式移项,反得$I_{k-1}=\dfrac{1-I_k}{k}$。——编者注}
    依次计算$I_7\rightarrow I_6\rightarrow\cdots\rightarrow I_0$。
\end{itemize}
实际上就整体而言,算法2精度更高。对算法1递推公式$I_k=1-kI_{k-1}$。若$I_{k-1}$有舍入误差$\Delta I_{k-1}$(或记作$\Delta$),则$\tilde{I}_k=1-k(I_{k-1}+\Delta)=I_k-k\cdot\Delta$,误差被放大
\footnote{对算法2的递推公式做类似分析,可见$\tilde{I}_{k-1}=I_{k-1}-\Delta/k$,即误差被减小到原来的$1/k$倍,这是大大缩小了。故算法2较算法1更为稳定。——编者注}。

\chapter{插值法}
\section{插值法思想概要}
\entry 插值的\emph{动机}:
\begin{enumerate}\tl
    \item 离散型:给定$m+1$个数据点$(x_0,y_0)$、$(x_1,y_1)$、……$(x_m,y_m)$,找到一个函数$P(x)$通过所有的点。
    \item 连续型:给定一未知函数$f(x)$及其$m+1$个数据点,要求另一已知函数$P(x)$在这
    些数据点上与$f(x)$一致,并使其在定义与上与$f(x)$的偏差尽量小。(称$f(x)$为\key{被插函数}。)
\end{enumerate}

\define 设以上的$P(x)$满足
\begin{equation}\label{4-e1}
P(x_i)=y_i\quad(i=0,1,\cdots,m),
\end{equation}
则称$P(x)$为这$m+1$个数据点上的\key{插值函数},称数据点为\key{插值点},并称式
(\ref{4-e1})为\key{插值条件}。

\entry \emph{多项式插值}:要求插值函数$P(x)$为一个多项式
\begin{equation}
P(x)=\sum_{k=0}^na_kx^k=a_0+a_1x+\cdots+a_nx^n,
\end{equation}
则称此时的$P(x)$为一个\key{插值多项式},对应的插值条件
\begin{equation}
P(x_i)=y_i\sothat a_0+a_1x_i+\cdots+a_ix_i^n=y_i\quad(i=0,1,\cdots,m)
\end{equation}
可视为一个线性方程组:
\begin{equation}\label{4-e2}
\begin{pmatrix}1&x_0&\cdots&x_0^n\\1&x_1&\cdots&x_1^n\\\vdots&\vdots&&\vdots\\
1&x_m&\cdots&x_m^n\end{pmatrix}\begin{pmatrix}a_0\\a_1\\\vdots\\a_n\end{pmatrix}
=\begin{pmatrix}y_0\\y_1\\\vdots\\y_n\\\end{pmatrix},
\end{equation}

\entry 关于方程组(\ref{4-e2})的解分析:
\begin{itemize}\tl
    \item $m>n$,方程一般无解;
    \item $m<n$,方程有无穷多解;
    \item $m=n$时,系数矩阵对应的行列式是 \emph{Van der monde行列式}:
    \begin{equation}
    |V|=\prod_{j=1}^n\left[\prod_{i=0}^{j-1}(x_j-x_i)\right]
    \end{equation}
    当$x_i\neq x_j\ (i\neq j)$时,$|V|\neq0$,方程组有唯一解。
\end{itemize}

\trm 对于给定的$n+1$个插值点,对应的$n$次插值多项式$P_n(x)$唯一存在。

\entry \key{误差多项式}:若被插函数$f(x)$满足$f^{(n)}(x)$在$[a,b]$上连续,$f^{(n+1)}$在
$(a,b)$内存在,则\emph{误差(多项式)}可估计为
\begin{equation}
R_n(x)=f(x)-P_n(x)=\frac{f^{(n+1)}(\xi)}{(n+1)!}(x-x_0)(x-x_1)\cdots(x-x_n).
\end{equation}
记
\begin{equation}
\pi_{n+1}(x)=(x-x_0)(x-x_1)\cdots(x-x_n),
\end{equation}
则有
\begin{equation}
R_n(x)=\frac{f^{(n+1)}(\xi)}{(n+1)!}\pi_{n+1}(x).
\end{equation}

\entry \emph{实用误差估计式}:设$P_n(x)$为在$(x_0,x_1,\cdots,x_n)$上的插值多项式,$P_n^\ast(x)$
为在$(x_0,x_1,\cdots,x_n,x_{n+1})$上的插值多项式,则
\begin{gather}
f(x)-P_n(x)\approx\frac{P_n^\ast(x)-P_n(x)}{x_{n+1}-x_0}(x-x_0)\\
f(x)-P_n^\ast(x)\approx\frac{P_n^\ast(x)-P_n(x)}{x_{n+1}-x_0}(x-x_{n+1})
\end{gather}

\entry 采用式(\ref{4-e2})求解插值多项式的问题:方程条件数随$n$的增加而急剧上升,解不
稳定、不精确;计算量太大。

\section{Lagrange 插值法与 Newton 插值法}
\entry \key{Lagrange 插值法}:对$n+1$个数据点,构造对应的$n+1$个\key{Lagrange插值基函数} $l_0(x),l_1(x),\cdots,l_n(x)$使
\begin{equation}\label{4-e3}
P_n(x)=y_0\cdot l_0(x)+y_1\cdot l_1(x)+\cdots+y_n\cdot l_n(x).
\end{equation}

\entry 插值基函数公式:由式(\ref{4-e3})易知,插值基函数应在数据点上满足
\begin{equation}
l_i(x_j)=\delta_{ij}=\begin{cases}1&,i=j\\0&,i\neq j\end{cases}
\end{equation}
又要求插值基函数为$n$次多项式(与最终插值多项式的次数一致),则可解出
\begin{gather}
l_i(x)=\frac{(x-x_0)(x-x_1)\cdots(x-x_{i-1})(x-x_{i+1})\cdots(x-x_n)}{(x_i-x_0)
(x_i-x_1)\cdots(x_i-x_{i-1})(x_i-x_{i+1})\cdots(x_i-x_n)}.
\end{gather}
此即\emph{Lagrange 插值基函数公式}。其分母$\pi_{n+1}(x)/(x-x_i)$可直接写出,故只需计算插值基函数的系数
\begin{equation}
c_i=\left[(x_i-x_0)(x_i-x_1)\cdots(x_i-x_{i-1})(x_i-x_{i+1})\cdots(x_i-x_n)\right]^{-1}.
\end{equation}

\example 给定数据点$(-1,7)$、$(1,7)$、$(2,4)$、$(5,35)$,用Lagrange插值法求解各插值基函数的系数。(答案:$c_0=-\frac1{36}$,$c_1=\frac18$,$c_2=-\frac19$,
$c_3=\frac1{72}$。)

\entry 常将Lagrange基函数记为
\begin{equation}
l_i(x)=\frac{\pi_{n+1}(x)}{(x-x_i)\pi'_{n+1}(x_i)}.
\end{equation}

\entry Lagrange插值法的特点:
\begin{itemize}\tl
    \item 构造方便、格式统一;
    \item 系数的\emph{表示}方法简单,但乘除\emph{运算}量大;
    \item 插值基函数具有\emph{全局性质},「牵一发而动全身」,数据点变动后须全部重新计算。
\end{itemize}

% P140 4.10:Lagrange 基函数的性质

\entry \key{Newton 插值法}:对$(n+1)$个数据点,采用「累进」的插值基函数
\begin{equation}
n_i(x)=\prod_{k=0}^{i-1}(x-x_k)=(x-x_0)(x-x_1)\cdots(x-x_{i-1}),
\end{equation}
构造得到的 \key{Newton 插值多项式}为
\begin{equation}\label{4-e4}
N_n(x)=c_0+c_1(x-x_0)+c_2(x-x_0)(x-x_1)+\cdots+c_n(x-x_0)\cdots(x-x_n)
\end{equation}
再求解系数$c_i$。

\entry \key{差商}:导数的一种离散形式,递归定义:
\begin{itemize}
    \item 零阶差商:$f[x_i]=f(x_i)=y_i\quad(i=0,1,\cdots,n)$.
    \item 一阶差商:$f[x_i,x_{i+1}]=\frac{f[x_{i+1}]-f[x_i]}{x_{i+1}-x_i}\quad
    (i=0,1,\cdots,n-1)$.
    \item 二阶差商:$f[x_i,x_{i+1},x_{i+2}]=\frac{f[x_i,x_{i+2}]-f[x_i,x_{i+1}]}{
    x_{i+2}-x_{i+1}}\quad(i=0,1,\cdots,n-2)$.
    \item ……
    \item $n$阶差商:$f[x_0,x_1,\cdots,x_n]=\frac{f[x_0,x_1,\cdots,x_{n-2},x_n]-
    f[x_0,x_1,\cdots,x_{n-2},x_{n-1}]}{x_n-x_{n-1}}$.
\end{itemize}

\entry 易证式(\ref{4-e4})中系数满足$c_0=f[x_0]$,$c_1=f[x_0,x_1]$,$c_2=f[x_0,x_1,
x_2]$……从而 Newton 插值多项式为
\begin{equation}
N_n(x)=\sum_{k=0}^nf[x_0,\cdots,x_k]\cdot n_k(x).
\end{equation}

\entry 由插值多项式的唯一性,对同样的$n+1$个数据点构造的 Lagrange 插值多项式$P_n(x)$
与 Newton 插值多项式$N_n(x)$,必有$P_n(x)=N_n(x)$,即\emph{两种方法得到的结果相同}。

\trm 推论:比较 Newton 插值多项式与 Lagrange 插值多项式的最高次($n$次)项系数,有
\begin{equation}
f[x_0,x_1,\ldots,x_n]=\sum_{i=0}^n\frac{f(x_i)}{\pi'_{n+1}(x_i)}.
\end{equation}

\trm 差商的性质:
\begin{enumerate}
    \item 差商仅与选取的具体点$(x_i,f(x_i))$有关,与它们的排列次序无关。
    \item $f[x_i,x_{i+1},\ldots,x_{i+k}]=f^{(k)}(\xi)/(k!)$,其中$\xi\in(\min\{x_i
    \},\max\{x_i\})$。
    \item 在以上结论中取$x_i=x_{i+1}=\ldots=x_{i+k}$,得
    \begin{equation}\label{4-e5}
    f[x_i,x_i,\ldots,x_i]=\frac{f^{(k)}(x_i)}{k!}.
    \end{equation}
    \item $\frac{\di f[x_0,x_1,\ldots,x_{k-1},x]}{\di x}=\frac{f^{(k+1)}(\xi)}{
    (k+1)!}$。
\end{enumerate}

\example 设$f(x)=x^3+2px+5qx+c$,其中$p,q,c$均为实数。若$f[1,2,m]=0$,试求$f[0,1,m]$。
(答案:$-2$)

\entry \key{差商表}:依次计算差商的工具,如图 \ref{4-f1} 所示。
\begin{figure}[htbp]
\small\centering
\[P_3(x)=y_0+y'_0\cdot(x-x_0)+y''_0\cdot(x-x_0)(x-x_1)+y'''_0\cdot(x-x_0)(x-x_1)(x-x_2)\]
\begin{tabular}{ccccc}
\toprule
$x_i$ & $f[x_i]$ & $f[x_i,x_{i+1}]$ & $f[x_i,x_{i+1},x_{i+2}]$ & $f[x_i,x_{i+1},x_{i+2},x_{i+3}]$ \\
\midrule
$x_0$ & \fbox{$y_0$} & & & \\
$x_1$ & $y_1$ & \fbox{$y'_0=\frac{y_1-y_0}{x_1-x_0}$} & & \\
$x_2$ & $y_2$ & $y'_1=\frac{y_2-y_1}{x_2-x_1}$ & \fbox{$y''_0=\frac{y'_1-y'_0}{x_2-x_0}$} & \\
$x_3$ & $y_3$ & $y'_2=\frac{y_3-y_2}{x_3-x_2}$ & $y''_1=\frac{y'_2-y'_1}{x_3-x_1}$ & \fbox{$y'''_0=\frac{y''_1-y''_0}{x_3-x_0}$} \\
\bottomrule
\end{tabular}
\caption{差商表及其构造步骤}\label{4-f1}
\end{figure}

\entry \key{Newton 插值法余项公式}及估计:
\begin{align*}
R_n(x)&=f(x)-N_n(x)\\
&=f[x_0,x_1,\ldots,x_n,x](x-x_0)(x-x_1)\cdots(x-x_n)\\
&\approx f[x_0,x_1,\ldots,x_n,x_{n+1}](x-x_0)(x-x_1)\cdots(x-x_n)\\
&=N_{n+1}(x)-N_n(x)
\end{align*}
即:Newton 插值公式$N_n(x)$的余项,可估计为高一阶的插值多项式$N_{n+1}(x)$之最后一项。

\entry \key{Hermite 插值多项式}:满足导数条件$P'(x_j)=y'_j$的插值多项式。

\entry Newton 插值法构造 Hermite 插值多项式(\key{重节点法}):在含导数条件的数据点处
增加「重节点」,仍按差商表迭代,但利用条件(\ref{4-e5})计算含重节点的差商。
\begin{figure}[htbp]
\small\centering
\begin{tabular}{ccccc}
    \toprule
    $x_i$ & $f[x_i]$ & $f[x_i,x_{i+1}]$ & $f[x_i,x_{i+1},x_{i+2}]$ & $f[x_i,x_{i+1},x_{i+2},x_{i+3}]$ \\
    \midrule
    $x_0$ & $y_0$ & & & \\
    $x_1$ & $y_1$ & $y'_0=\frac{y_1-y_0}{x_1-x_0}$ &  & \\
    $x_1$ & $y_1$ & \fbox{$y'_1$} & $y''_0=\frac{y'_1-y'_0}{x_1-x_0}$ & \phantom{$\dfrac12$} \\
    $x_2$ & $y_2$ & $y'_2=\frac{y_2-y_1}{x_2-x_1}$ & $y''_1=\frac{y'_2-y'_1}{x_2-x_1}$ & $y'''_0=\frac{y''_1-y''_0}{x_2-x_0}$ \phantom{$\dfrac12$} \\
    \bottomrule
    \end{tabular}
\caption{重节点法示意}\label{4-f2}
\end{figure}

\entry Hermite 插值多项式的误差估计:利用 Newton 插值的余项估计即可。

\example 对以下的数据点求解其 Hermite 插值多项式,并估计误差。
\begin{center}\small
\begin{tabular}{cccc}
\toprule
$x_i$&$-1$&$0$&$1$\\
\midrule
$y_i$&$0$&$-4$&$5$\\
$y'_i$&&$0$&$5$\\
$y''_i$&$6$&&\\
\bottomrule
\end{tabular}\end{center}
(答案:$H_5(x)=x^5-2x^3+3x^2-4$,
$R_5(x)=\frac{f^{(6)}(\xi)}{6!}(x+1)(x-1)^2x^3$。)

\entry Lagrange 插值法构造 Hermite 插值多项式:对$n+1$个含导数条件的数据点,在导数
条件下构造插值基函数$h_i(x)$与$\overline{h}_i(x)$:
\begin{gather}
H_{2n+1}(x)=\sum_{i=0}^nh_i(x)f(x_i)+\sum_{i=0}^n\overline{h}_i(x)f'(x_i)\\
h_i(x_j)=\delta_{ij},\quad h'_i(x_j)=0\quad\overline{h}_i(x_j)=0\quad
\overline{h'}_i(x_j)=\delta_{ij}.
\end{gather}
分析零点重数可推得
\begin{gather}
h_i(x)=(ax+b)l_i^2(x),\quad\overline{h}_i(x)=(x-x_i)l_i^2(x)
\end{gather}
再求解系数$a$与$b$即可。

\entry 不建议使用 Lagrange 插值法求解 Hermite 插值多项式:\emph{优势尽失}。

\section{分段插值与三次样条插值}
\entry \key{Runge 现象}:采用高次的插值多项式,全局误差可能比低次多项式更大。(示例:对函数 $f(x)=\frac1{1+25x^2}\quad(-1\leq x\leq 1)$ 以越来越多的节点等距插值。)这表明,与其在全局应用高次插值多项式,不如采用分段低次插值多项式。

\entry \key{分段线性插值}的误差估计:
\begin{equation}
R_{1,j}(x)=\left|\frac{f''(\xi)}2(x-x_{i-1})(x-x_i)\right|\leq\frac{M_2i}2\cdot
\frac14(x_i-x_{i-1})^2\leq\frac18M_2\Delta^2
\end{equation}
其中$M_2$为$f''(x)$在插值区间上的最大值,$\Delta$为各相邻插值节点的最大距离。

\begin{figure}[htbp]
\small\centering
\begin{tikzpicture}
\draw[->] (-0.2,0) -- (4.5,0) node[right] {$x$};
\draw[->] (0,-0.2) -- (0,2.5) node[above] {$y$};
\draw plot[mark=ball,ball color=black] coordinates {(0.5,1) (1,2) (2,1) (3,0.5) (4,0.3)};
\end{tikzpicture}
\caption{分段线性插值图示}\label{4-f3}
\end{figure}

\entry 分段线性插值的\emph{缺陷}:在各个节点处导数不连续。

\entry \key{分段二次插值}:在$[x_{i-1},x_{i+1}]$上作 Newton 二次插值多项式$N_{2i}(x)$,则
\begin{equation}
f(x)\approx N_{2i}(x)=y_{i-1}+f[x_{i-1},x_i](x-x_{i-1})+f[x_{i-1},x_i,x_{i+1}]
(x-x_{i-1})(x-x_i).
\end{equation}

\entry 分段二次插值的误差估计:
\begin{equation}
R_{2i}(x)=\frac{f'''(\xi)}{3!}(x-x_{i-1})(x-x_i)(x-x_{i+1})\leq\frac{M_3}6\cdot
\frac14\Delta^2\cdot2\Delta=\frac1{12}M_3\Delta^3
\end{equation}
其中$M_3$为$f'''(x)$在插值区间上的最大值,$\Delta$为各相邻插值节点的最大距离。

\entry 分段二次插值的缺陷:在一半的节点上导数仍不连续;要求有奇数个插值节点。

\begin{figure}[htbp]
\small\centering
\begin{tikzpicture}
\draw[->] (-0.2,0) -- (4.5,0) node[right] {$x$};
\draw[->] (0,-0.2) -- (0,2.5) node[above] {$y$};
\draw (0.5,1) parabola bend (1.25,2.125) (2,1);
\draw (2,1) parabola[bend at end] (4,0.3);
\draw[white] plot[mark=ball,ball color=black] coordinates {(0.5,1) (1,2) (2,1) (3,0.5) (4,0.3)};
\end{tikzpicture}
\caption{分段二次插值图示}\label{4-f4}
\end{figure}

\entry \key{分段三次 Hermite 插值}:在相邻插值节点处利用两个函数值、两个导数值构造插值多项式。

\entry 分段三次 Hermite 插值的误差估计:
\begin{equation}
R_{3,i}(x)=\frac{f^{(4)}(\xi)}{4!}(x-x_{i-1})^2(x-x_i)^2\leq\frac{M_4}{24}
\left[\frac14(x-x_{i-1})^2\right]^2\leq\frac{M_4}{384}\Delta^4.
\end{equation}

\entry 分段三次 Hermite 插值的缺陷:二阶导数仍不连续;导数条件太苛刻,插值时一般缺少相应导数。

\entry \key{三次样条插值}:给定$n+1$个数据点,要求分段插值曲线的二阶导数连续。
由此可知,插值函数及其导数也应连续。此时:
\begin{itemize}\tl
    \item \emph{需求}:在$n$个插值子区间上构造\emph{分段三次插值函数},共计$4n$个未知数;
    \item \emph{约束条件}:$n+1$个数据点,$3(n-1)$个各阶导数连续条件,共计$4n-2$个方程;
    \item 补充 2 个\emph{边界条件}:一阶导数边界条件$f'(a)$、$f'(b)$,或二阶导数边界条件
    $f'(a)$、$f'(b)$,或周期性边界条件$f'(a)=f'(b)$、$f''(a)=f''(b)$。
\end{itemize}
由此即可求解出所有的分段三次插值函数,称这些函数为\key{三次样条函数}。

\entry \key{三弯矩方程}:设 $S(x)$ 在 $x_i$ 处二阶导数值为 $M_i$,根据插值函数为三次可知 $S''(x)$ 为连续的分段线性函数,对 $S''(x)$ 积两次分并用 $n$ 个数据点条件消去未知参数得:
\begin{equation}\label{4-e7}\begin{aligned}
S(x)=&\frac{(x_i-x)^3}{6h_i}M_{i-1}+\frac{(x-x_{i-1})^3}{6h_i}M_i+\left(y_{i-1}-\frac{h_i^2}6M_{i-1}\right)\frac{x_i-x}{h_i}\\
&+\left(y_i-\frac{h_i^2}6M_i\right)\frac{x-x_{i-1}}{h_i},\ x_{i-1}\leq x\leq x_i
\end{aligned}\end{equation}
为求出 $M_i$,可对 $S(x)$ 在不同区间上的表达式求导,并应用 $(n-1)$ 个一阶导数连续条件获得 $(n-1)$ 个方程:
\begin{equation}\label{4-e6}
\mu_iM_{i-1}+2M_i+\lambda M_{i+1}=d_i,\quad i=1,2,\cdots,n-1
\end{equation}
其中
\[
\mu_i=\frac{h_i}{h_i+h_{i+1}}.,\quad\lambda_i=\frac{h_{i+1}}{h_i+h_{i+1}}=1-\mu_i,\quad d_i=6f[x_{i-1},x_i,x_{i+1}]
\]
若将 $S(x)$ 视为一根梁的挠度,则以上的 $M_i$ 可视为作用在 $x_i$ 处的弯矩,故称方程 \eqref{4-e6} 为\emph{三弯矩方程}。

\entry 三种边界条件下的三弯矩方程:三弯矩方程不足以解出 $n+1$ 个未知量,补充两个边界条件后方程即可封闭。
\begin{itemize}
    \item \emph{一阶导数边条}:对式 \eqref{4-e7} 求一次导,并代入边界导数 $f'(a)$ 与 $f'(b)$,可最终获得两个补充方程:
    \begin{equation}
    2M_0+M_1=d_0,\quad M_{n-1}+2M_n=d_n
    \end{equation}
    其中 $d_0$ 与 $d_n$ 与其他 $d_i$ 的定义一致。方程组变为 $n+1$ 个式子的
    \begin{equation}
    \begin{cases}
    2M_0+M_1=d_0\\
    \mu_iM_{i-1}+2M_i+\lambda_iM_{i+1}=d_i,\quad i=1,2,\ldots,n-1\\
    M_{n-1}+2M_n=d_n
    \end{cases}
    \end{equation}
    此时的系数矩阵是\emph{严格三对角占优矩阵}($\mu_i+\lambda_i=1<2$),可用\emph{追赶法}快速求解。
    \item \emph{二阶导数边条}:$M_0=f''(a)$ 和 $M_n=f''(b)$ 给定,方程组变为 $n-1$ 个式子的
    \begin{equation}
    \begin{cases}
    2M_1+\lambda_1M_2=d_1-\mu_1M_0\\
    \mu_iM_{i-1}+2M_i+\lambda_iM_{i+1}=d_i,\quad i=2,3,\ldots,n-2\\
    \mu_{n-1}M_{n-2}+2M_{n-1}=d_{n-1}-\lambda_{n-1}M_n
    \end{cases}
    \end{equation}
    此时系数矩阵也是\emph{严格三对角占优矩阵}。
    \item \emph{周期性边条}:根据周期性
    \footnote{此时 $M_0=M_n$,同时少去一个未知数和一个方程,仍需补充两个方程。}
    ,从边界点处「回环」得满足三弯矩方程的序列 $(M_{n-1},M_n,M_1)$ 与 $(M_n,M_1,M_2)$,由此补充两个方程,方程组变为 $n$ 个式子的:
    \begin{equation}
    \begin{cases}
    2M_1+\lambda_1M_2+\mu_1M_n=d_1\\
    \mu_iM_{i-1}+2M_i+\lambda_iM_{i+1}=d_i,\quad i=2,3,\ldots,n-1\\
    \lambda_nM_1+\mu_nM_{n-1}+2M_n=d_n
    \end{cases}
    \end{equation}
    此时的系数矩阵为\emph{严格对角占优矩阵}(不是三对角),可用一般的 LU 分解求解
    \footnote{在 LU 分解的过程中,会发现此形式的系数矩阵有与追赶法类似的简化算法,参见李乃成、梅立泉《数值分析》习题 2.4。}
    。
\end{itemize}

\entry 三次样条插值的\emph{特点}:
\begin{itemize}
    \item [$\sqrt{}$] 求解的三弯矩方程为严格对角占优矩阵,方程形式简洁且容易求解,解存在唯一、稳定性好;
    \item [$\sqrt{}$] 为提高精度,只需增加插值节点,不需要提高次数;
    \item [$\sqrt{}$] 随节点增多,$S(x)$ 及其一二阶导数一致收敛于 $f(x)$ 及其对应导数;若节点间等距,则 $S'''(x)$ 随节点加密而一致收敛于 $f'''(x)$。
    \item [$\times$] 三弯矩方程计算量仍很大;
    \item [$\times$] 需要额外的边界条件,有时难以获得;
    \item [$\times$] 对图形的控制不够灵活,绘图上使用并不方便
    \footnote{绘图上常用 B\'ezier 曲线、B 样条曲线等专用于拟合几何边界的插值/曲线构造方式。}
    。
\end{itemize}
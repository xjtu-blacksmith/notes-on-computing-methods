% 宏包补充:索引及子图
\usepackage{makeidx}
\usepackage{subcaption}

% 条目计数器
\newcounter{entry}
\counterwithin*{entry}{chapter}

% 条目命令
\newcommand{\example}{\vskip1ex\refstepcounter{entry}\noindent$\text{\ding{48}}{}_{\makebox%
[10pt]{\footnotesize\bf\arabic{chapter}.\arabic{entry}}\hspace{5pt}}$}
\newcommand{\entry}{\vskip1ex\refstepcounter{entry}\noindent$\text{\ding{226}}_{\makebox%
[10pt]{\footnotesize\bf\arabic{chapter}.\arabic{entry}}\hspace{5pt}}$}
\newcommand{\trm}{\vskip1ex\refstepcounter{entry}\noindent$\text{\ding{43}}_{\makebox%
[10pt]{\footnotesize\bf\arabic{chapter}.\arabic{entry}}\hspace{5pt}}$}
\newcommand{\define}{\trm}

% 紧列表命令
\newcommand{\tl}{\setlength{\itemsep}{0pt}\setlength{\parskip}{0pt}}

% 符号简写及算子声明
\newcommand{\e}{\mathrm{e}}
\newcommand{\di}{\mathrm{d}}
\newcommand{\del}{\partial}
\newcommand{\sothat}{\ \Rightarrow\ }
\newcommand{\vphi}{\varphi}
\newcommand{\vepsilon}{\varepsilon}
\newcommand{\dsum}{\sum\limits}
\newenvironment{spmatrix}{\left(\begin{smallmatrix}}{\end{smallmatrix}\right)}
\DeclareMathOperator{\diag}{diag}
\DeclareMathOperator{\cond}{Cond}
\DeclareMathOperator{\sgn}{sgn}

% 字体及索引格式
\newcommand{\add}[1]{\textsf{[#1]}}
\newcommand{\key}[1]{\textbf{#1}\index{#1}}
\newcommand{\setkey}[2]{\textbf{#1}\index{#2}}
\renewcommand{\emph}[1]{\uline{#1}}
